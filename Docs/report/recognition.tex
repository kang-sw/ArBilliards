\documentclass[11pt]{oblivoir}
\usepackage{kotex}
\usepackage{graphicx} 
\usepackage{fullpage}

\title{AR Billiards}
\author{한국기술교육대학교 전자공학과 강승우}
\date{2020.09.24}

\begin{document}
\maketitle
\tableofcontents

\begin{abstract}
당구에 처음 입문하는 초심자가 테이블 위에서 벌어지는 물리 법칙에 적응하는 데에는 꽤 오랜 시간이 걸리는데, 어른 아이 할 것 없이 주머니 속에 고성능 오락 기기를 넣고 다니는 시대에 즐기는 데에도 오랜 연습을 필요로 하는 당구는 손쉽게 다가가기 어려운 스포츠이다. 

AR 당구는 입문자들이 당구의 높은 진입장벽을 극복할 수 있도록 만들어졌다. 사용자에게 몰입감 있는 경험을 선사할 수 있는 VR HMD\footnote{Head Mounted Display의 약자; VR 헤드셋과 같이 머리에 장착하여 사용자의 시야 전체를 채우는 형태의 장착형 디스플레이 기기를 일컬음} 상에서 당구공이 득점 가능한 최적의 경로 및 사용자가 취한 큐의 각도에 대한 피드백을 시각화하고, 여러 시청각 효과를 활용한 오락 요소를 도입하여 입문자의 흥미 유발과 당구에 대한 감각 습득을 돕는다.

AR 당구는 다음의 세 가지 요소로 이루어진다; (1) 영상 처리를 통한 당구대, 당구공, 큐의 3D Pose 획득 (2) 득점 경로를 계산하기 위한 물리 시뮬레이션 구현 (3) 오브젝트의 3D 포즈 및 계산된 득점 경로 등에 대한 시각화.

(Abstract 내 최종 구현 내용은 추후 기술)
\end{abstract}

\section{영상 처리}
\subsection{개요}

\subsection{테이블 인식}
\subsubsection{테이블의 일부만 시야에 들어온 경우}

\subsection{당구공 인식}

\subsection{큐 인식}

\section{물리 시뮬레이션}

\end{document}
